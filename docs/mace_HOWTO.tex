% 
% mace_HOWTO.tex : part of the Mace toolkit for building distributed systems
% 
% Copyright (c) 2007, Charles Killian, Dejan Kostic, Ryan Braud, James W. Anderson, John Fisher-Ogden, Calvin Hubble, Duy Nguyen, Justin Burke, David Oppenheimer, Amin Vahdat, Adolfo Rodriguez, Sooraj Bhat
% All rights reserved.
% 
% Redistribution and use in source and binary forms, with or without
% modification, are permitted provided that the following conditions are met:
% 
%    * Redistributions of source code must retain the above copyright
%      notice, this list of conditions and the following disclaimer.
%    * Redistributions in binary form must reproduce the above copyright
%      notice, this list of conditions and the following disclaimer in
%      the documentation and/or other materials provided with the
%      distribution.
%    * Neither the names of Duke University nor The University of
%      California, San Diego, nor the names of the authors or contributors
%      may be used to endorse or promote products derived from
%      this software without specific prior written permission.
% 
% THIS SOFTWARE IS PROVIDED BY THE COPYRIGHT HOLDERS AND CONTRIBUTORS "AS IS"
% AND ANY EXPRESS OR IMPLIED WARRANTIES, INCLUDING, BUT NOT LIMITED TO, THE
% IMPLIED WARRANTIES OF MERCHANTABILITY AND FITNESS FOR A PARTICULAR PURPOSE ARE
% DISCLAIMED. IN NO EVENT SHALL THE COPYRIGHT OWNER OR CONTRIBUTORS BE LIABLE
% FOR ANY DIRECT, INDIRECT, INCIDENTAL, SPECIAL, EXEMPLARY, OR CONSEQUENTIAL
% DAMAGES (INCLUDING, BUT NOT LIMITED TO, PROCUREMENT OF SUBSTITUTE GOODS OR
% SERVICES; LOSS OF USE, DATA, OR PROFITS; OR BUSINESS INTERRUPTION) HOWEVER
% CAUSED AND ON ANY THEORY OF LIABILITY, WHETHER IN CONTRACT, STRICT LIABILITY,
% OR TORT (INCLUDING NEGLIGENCE OR OTHERWISE) ARISING IN ANY WAY OUT OF THE
% USE OF THIS SOFTWARE, EVEN IF ADVISED OF THE POSSIBILITY OF SUCH DAMAGE.
% 
% ----END-OF-LEGAL-STUFF----
\documentclass[10pt]{article}

\setlength{\textwidth}{6.50in}
\setlength{\topmargin}{0.0in}
\setlength{\textheight}{9.0in}
\setlength{\headheight}{0.0in}
\setlength{\headsep}{0.0in}
\setlength{\oddsidemargin}{0in}
\setlength{\parindent}{1em}
%% \setlength{\parskip}{1.5ex plus0.5ex minus 0.5ex}

\usepackage{ifthen}
\usepackage{times}
\usepackage{url}
\usepackage{xspace}
\usepackage{verbatim}
\usepackage{fancyvrb}
\usepackage{alltt}
\usepackage{epsfig}
\usepackage[lineno5]{lgrind}
%%%%%%%%%%%%%%%%%%%%%%%%%%%%%%%%%%%%%%%%%%%%%%%%%%%%%%%%%%%%%%%%%%%%%%%%
% hyperref package
%%%%%%%%%%%%%%%%%%%%%%%%%%%%%%%%%%%%%%%%%%%%%%%%%%%%%%%%%%%%%%%%%%%%%%%%
\ifx\pdfoutput\undefined %%% for ps2pdf
\usepackage[ps2pdf,
bookmarks=true,
linkbordercolor={0 0 1}]{hyperref}
\else %%% we are using pdftex
\usepackage[pdftex,
bookmarks=true,
%% hypertexnames=false,
breaklinks=true,
linkbordercolor={0 0 1}]{hyperref}
\fi %%% endif for hyperref

\newcommand{\mac}{\filename{.(m|mac|mace)}}
\newcommand{\authorlist}{James W. Anderson, Charles Killian, and Amin Vahdat}

\newcommand{\filename}{\texttt}
\newcommand{\directory}{\texttt}
\newcommand{\classname}{\textit}
\newcommand{\typename}{\textit}
\newcommand{\literal}{\textit}
\newcommand{\variablename}{\texttt}
\newcommand{\function}{\texttt}
\newcommand{\command}{\textbf}
\newcommand{\replaceable}{\textit}
\newcommand{\symbolkw}{\texttt}

\newenvironment{screen}{\begin{alltt}}{\end{alltt}}
\DefineVerbatimEnvironment{programlisting}{Verbatim}{frame=lines,fontsize=\scriptsize,
  numbers=right,numbersep=1mm,framesep=2mm}

\title{Mace + Distributed Systems HOWTO}

\author{James W. Anderson\\\emph{jwanderson@cs.ucsd.edu} \and
  Charles Killian\\\emph{ckillian@cs.ucsd.edu} \and
  Amin Vahdat\\\emph{vahdat@cs.ucsd.edu}}

\begin{document}

\maketitle

\bigskip
\begin{quote}
Copyright \copyright 2005, \authorlist.

All rights reserved.

Terms of redistribution are described in \S~\ref{sec:copyright}, but 
follow a BSD-Style license.
\end{quote}
\bigskip

\begin{abstract}
\noindent
Mace is a programming language and toolkit for building robust and
high-performance distributed systems.  The Mace programming language
provides constructs to specify services, protocols, and distributed
systems as a composition of services, messages, and C++ code.  The
Mace toolkit provides implementations for many classes of services,
including routing, overlay routing, multicast, aggregation, HTTP, and
distributed hash tables.  The Mace toolkit also includes libraries for
encryption, string processing, file I/O, serialization, and XML-RPC.
\end{abstract}

\tableofcontents

% 
% preamble.tex : part of the Mace toolkit for building distributed systems
% 
% Copyright (c) 2007, Charles Killian, Dejan Kostic, Ryan Braud, James W. Anderson, John Fisher-Ogden, Calvin Hubble, Duy Nguyen, Justin Burke, David Oppenheimer, Amin Vahdat, Adolfo Rodriguez, Sooraj Bhat
% All rights reserved.
% 
% Redistribution and use in source and binary forms, with or without
% modification, are permitted provided that the following conditions are met:
% 
%    * Redistributions of source code must retain the above copyright
%      notice, this list of conditions and the following disclaimer.
%    * Redistributions in binary form must reproduce the above copyright
%      notice, this list of conditions and the following disclaimer in
%      the documentation and/or other materials provided with the
%      distribution.
%    * Neither the names of Duke University nor The University of
%      California, San Diego, nor the names of the authors or contributors
%      may be used to endorse or promote products derived from
%      this software without specific prior written permission.
% 
% THIS SOFTWARE IS PROVIDED BY THE COPYRIGHT HOLDERS AND CONTRIBUTORS "AS IS"
% AND ANY EXPRESS OR IMPLIED WARRANTIES, INCLUDING, BUT NOT LIMITED TO, THE
% IMPLIED WARRANTIES OF MERCHANTABILITY AND FITNESS FOR A PARTICULAR PURPOSE ARE
% DISCLAIMED. IN NO EVENT SHALL THE COPYRIGHT OWNER OR CONTRIBUTORS BE LIABLE
% FOR ANY DIRECT, INDIRECT, INCIDENTAL, SPECIAL, EXEMPLARY, OR CONSEQUENTIAL
% DAMAGES (INCLUDING, BUT NOT LIMITED TO, PROCUREMENT OF SUBSTITUTE GOODS OR
% SERVICES; LOSS OF USE, DATA, OR PROFITS; OR BUSINESS INTERRUPTION) HOWEVER
% CAUSED AND ON ANY THEORY OF LIABILITY, WHETHER IN CONTRACT, STRICT LIABILITY,
% OR TORT (INCLUDING NEGLIGENCE OR OTHERWISE) ARISING IN ANY WAY OUT OF THE
% USE OF THIS SOFTWARE, EVEN IF ADVISED OF THE POSSIBILITY OF SUCH DAMAGE.
% 
% ----END-OF-LEGAL-STUFF----
\section{Preamble}
\label{sec:preamble}
%Credits, License, etc.

\subsection{Copyrights and Distribution}
\label{sec:copyright}

This document is released under ``the new BSD license'' (listed below).  The
Mace software package is distributed under the same license restrictions.  This
manual contains short example programs (``the Software'').  Permission is
hereby granted, free of charge, to any person obtaining a copy of the Software,
to deal in the Software without restriction, including without limitation the
rights to use, copy, modify, merge, publish, distribute, sublicense, and/or
sell copies of the Software, and to permit persons to whom the Software is
furnished to do so, subject to the conditions of the license:

%% The Mace software package will be distributed under the same license
%% once a public release is made, until then it is for internal use only,
%% and redistribution in any form is strictly prohibited.

\bigskip
\begin{quote}
Copyright \copyright 2008, \authorlist.

All rights reserved.

Redistribution and use in source and binary forms, with or without
modification, are permitted provided that the following conditions are met:

\begin{itemize}
\item Redistributions of source code must retain the above copyright notice, this
  list of conditions and the following disclaimer.
\item Redistributions in binary form must reproduce the above copyright notice,
this list of conditions and the following disclaimer in the documentation
and/or other materials provided with the distribution.
\item Neither the name of the University of California, San Diego, nor the
names of its contributors may be used to endorse or promote products derived
from this software without specific prior written permission.
\end{itemize}

THIS SOFTWARE IS PROVIDED BY THE COPYRIGHT HOLDERS AND CONTRIBUTORS "AS IS" AND
ANY EXPRESS OR IMPLIED WARRANTIES, INCLUDING, BUT NOT LIMITED TO, THE IMPLIED
WARRANTIES OF MERCHANTABILITY AND FITNESS FOR A PARTICULAR PURPOSE ARE
DISCLAIMED. IN NO EVENT SHALL THE COPYRIGHT OWNER OR CONTRIBUTORS BE LIABLE FOR
ANY DIRECT, INDIRECT, INCIDENTAL, SPECIAL, EXEMPLARY, OR CONSEQUENTIAL DAMAGES
(INCLUDING, BUT NOT LIMITED TO, PROCUREMENT OF SUBSTITUTE GOODS OR SERVICES;
LOSS OF USE, DATA, OR PROFITS; OR BUSINESS INTERRUPTION) HOWEVER CAUSED AND ON
ANY THEORY OF LIABILITY, WHETHER IN CONTRACT, STRICT LIABILITY, OR TORT
(INCLUDING NEGLIGENCE OR OTHERWISE) ARISING IN ANY WAY OUT OF THE USE OF THIS
SOFTWARE, EVEN IF ADVISED OF THE POSSIBILITY OF SUCH DAMAGE.
\end{quote}

\subsection{Acknowledgments}
\label{sec:acknowledgements}

We thank Adolfo Rodriguez and Dejan Kostic for making Mace possible by
developing the original system for building overlay networks, MACEDON.
We also thank all the other contributors who have worked on Mace or
MACEDON, including Ryan Braud, Darren Dao, Alex Rasumssen, Jim Hong,
John Fisher-Ogden, Calvin Hubble, Jeannie Albrecht, Duy Nguyen, Justin
Burke, David Oppenheimer, Sooraj Bhat, Ranjit Jhala, and Amin Vahdat.
We thank as well in advance all reviewers of this document and those who
give feedback.

\subsection{Feedback}
\label{sec:feedback}

Please send feedback and corrections to Charles Killian
(\href{mailto:ckillian@cs.ucsd.edu}{ckillian@cs.ucsd.edu}), James W. Anderson
(\href{mailto:jwanderson@cs.ucsd.edu}{jwanderson@cs.ucsd.edu}).  We will be
grateful for corrections and will incorporate them as soon as possible.

% 
% intro.tex : part of the Mace toolkit for building distributed systems
% 
% Copyright (c) 2007, Charles Killian, Dejan Kostic, Ryan Braud, James W. Anderson, John Fisher-Ogden, Calvin Hubble, Duy Nguyen, Justin Burke, David Oppenheimer, Amin Vahdat, Adolfo Rodriguez, Sooraj Bhat
% All rights reserved.
% 
% Redistribution and use in source and binary forms, with or without
% modification, are permitted provided that the following conditions are met:
% 
%    * Redistributions of source code must retain the above copyright
%      notice, this list of conditions and the following disclaimer.
%    * Redistributions in binary form must reproduce the above copyright
%      notice, this list of conditions and the following disclaimer in
%      the documentation and/or other materials provided with the
%      distribution.
%    * Neither the names of Duke University nor The University of
%      California, San Diego, nor the names of the authors or contributors
%      may be used to endorse or promote products derived from
%      this software without specific prior written permission.
% 
% THIS SOFTWARE IS PROVIDED BY THE COPYRIGHT HOLDERS AND CONTRIBUTORS "AS IS"
% AND ANY EXPRESS OR IMPLIED WARRANTIES, INCLUDING, BUT NOT LIMITED TO, THE
% IMPLIED WARRANTIES OF MERCHANTABILITY AND FITNESS FOR A PARTICULAR PURPOSE ARE
% DISCLAIMED. IN NO EVENT SHALL THE COPYRIGHT OWNER OR CONTRIBUTORS BE LIABLE
% FOR ANY DIRECT, INDIRECT, INCIDENTAL, SPECIAL, EXEMPLARY, OR CONSEQUENTIAL
% DAMAGES (INCLUDING, BUT NOT LIMITED TO, PROCUREMENT OF SUBSTITUTE GOODS OR
% SERVICES; LOSS OF USE, DATA, OR PROFITS; OR BUSINESS INTERRUPTION) HOWEVER
% CAUSED AND ON ANY THEORY OF LIABILITY, WHETHER IN CONTRACT, STRICT LIABILITY,
% OR TORT (INCLUDING NEGLIGENCE OR OTHERWISE) ARISING IN ANY WAY OUT OF THE
% USE OF THIS SOFTWARE, EVEN IF ADVISED OF THE POSSIBILITY OF SUCH DAMAGE.
% 
% ----END-OF-LEGAL-STUFF----
\section{Why are you reading this document?}
\label{sec:intro}

Perhaps you stumbled on this document on the web somewhere, perhaps
you have already downloaded Mace and are trying to figure out how to
use it, perhaps you are considering whether to use Mace, or perhaps
you've been told to use Mace by a boss or instructor, and need to
figure out what that means.

This document will explain what Mace is, go through a simple example
of how to use it, and discuss some of its advanced features as well as
address common concerns and questions.

\subsection{What is Mace?}
\label{sec:what-is-mace}

If you found this document looking for information on various forms of
bludgeoning weapons or self-defense sprays, then this paragraph will
be all that is in this document about your search.  None of these are
the context in which Mace will be discussed in this document.  The
name Mace is instead a derivative from the MACEDON parent project,
abbreviated to suggest the broader scope to which it is applicable.

Mace is a software package for building distributed systems.  It
builds upon the ideas from its parent project, MACEDON, by broadening
the scope of what can be designed with it, and by removing many
limitations of the original system.

Mace includes a compiler that translates service specifications into
C++ code, libraries designed to be linked together with generated
services, a distribution of existing services ready to be used by
other services or applications, and a few basic applications to run
the services contained within.

Mace seeks to transform the way distributed systems are built by
providing designers with a simple method for writing complex but
correct and efficient implementations of distributed systems.  To that
end, we are always considering new libraries and language features
which could be used to make building, designing, debugging, or
verifying distributed systems more powerful, flexible, simple, or
natural.

\subsection{Why would I want to use a domain-specific language and toolkit?}
\label{sec:why-new-toolkit}

Over the past few years, we have set out to evaluate a variety of
techniques for building robust, high-performance distributed systems.
One of our explicit goals has been to conduct head-to-head comparisons
of algorithms proposed in the literature, including our own.  Target
systems include application-layer multicast, distributed hash trees,
peer-to-peer indexing systems, and overlay routing.  One of the
primary lessons of this work thus far has been the difficulty of
translating elegant and seemingly simple distributed algorithms into
operational, high-performance, and robust systems.

We found significant commonality among the various
implementations---for instance, event handling, timer management,
failure detection, asynchronous communication, node join/departure, and
message serialization/deserialization.  However, differences in
individual application structure and implementation forced us to
repeatedly reimplement the same functionality, typically with a
different set of errors and inefficiencies each time.  Even once the
baseline system was operational, the resulting application often
performed significantly slower than expected and bugs in corner cases
would remain dormant or masked for long periods of time.

We concluded that a significant impediment to rapid prototyping of our
target applications was the lack of an appropriate development
environment for distributed applications.  Based on our experience, we
identified the following requirements to ease this development effort.

%\begin{itemize}
%\item Constructing distributed systems would be simplified by the
%  ability to compose simple distributed computing primitives into more
%  complex behavior.  
\emph{Constructing distributed systems would be simplified by the
ability to compose simple distributed computing primitives into more
complex behavior.} For instance, many distributed applications would
benefit from failure detection, consensus, multicast, barriers, and
key-based routing.  However, without well-defined API's, it is
difficult to reuse implementations or to leverage the benefits of an
improved implementation of a given logical subsystem.  In this paper,
we describe initial efforts to define required API's to support
complex, multi-layer distributed systems.

  %\item Current programming languages are not well suited to the
  %  requirements of distributed systems.  

\emph{Current programming languages are not well suited to the
requirements of distributed systems.} While there are communication
libraries and class hierarchies in languages such as C++, Java, and
Python, they typically target client/server communications (e.g., HTTP
or XMLRPC) and still provide relatively primitive support for failure
detection and recovery.  Further, we observe that the higher-level
structure of many distributed systems is logically event- and
state-based.  Each node maintains some state that may be modified as a
result of a series of events, typically message reception and timer
expiration.  Individual nodes respond to events by modifying their
state and perhaps transmitting their own message to one or more
destinations.  While this high level structure is simple to describe,
it is error prone to implement.  Further, managing asynchrony still
remains a challenge.  Delivering high performance often requires
careful consideration of appropriate locking primitives, ensuring that
individual operations do not block, and assigning the appropriate
number of threads to handle logically concurrent tasks.  Of course,
all of this can be programmed in existing languages such as C++ and,
to a lesser extent, Java. Providing the appropriate language
primitives can both significantly simplify the code and reduce
opportunities for errors.

  %\item Debugging support for distributed systems remains primitive,
  %  often reducing to custom scripts run over log files generated at
  %  hundreds of individual sites.  

\emph{Debugging support for distributed systems remains primitive,
often reducing to custom scripts run over log files generated at
hundreds of individual sites.}  Generating the necessary debugging
information can make the underlying code less efficient, less
readable, and more error prone.  With appropriate language support,
the appropriate logging information can be generated
semi-automatically.  Further, the information can be efficiently
stored in a format amenable to inserting into a SQL database.  We have
found this log of per-event system activity along with separate
programmer annotations for their \emph{expectations} of how the
distributed system should behave and \emph{assertions} about global
correctness conditions (both of which must necessarily simultaneously
apply to state stored at multiple nodes across at multiple time
granularities) to be invaluable in debugging distributed system
behavior.  These distributed expectations and assertions may apply to
to both system performance and distributed system structure.
%\end{itemize}
 
We are building Mace to be just the support that a designers and developers
need when building distributed systems.  It is a complete redesign and rewrite
of its parent project, MACEDON, targeted at a broader range of distributed
systems, and with better support for the programmer with fewer limitations.

\subsection{Why shouldn't I just use other previously existing languages?}
\label{sec:why-mace}

Though there are a few existing languages with similar goals, most do
not have the emphasis on building real systems as Mace does.  MACEDON
of course is very close, but suffers from limitations such as fixed
size message headers, restricted and monolithic API, non-portable 
network communications, lack of support for connecting with external
processes (such as XML-RPC or HTTP), lack of support for firewall
traversal, more primitive logging, and a single linear protocol/service
stack.

(Ed. Note: Comments on other systems forthcoming).

%Think of Mace as being for distributed systems what XSLT is
%for web documents.  Of course, you can write your web pages directly
%in the presentation language (HTML), but if you write them in a
%language which allows you to give meaning to the document (XML), they
%can be processed more effectively by automated tools, allowing them to
%be presented in different ways, analyzed for semantic meaning, or
%updated and modified.

%% av:
%% This is a great start at the answer to this question.

%% However, I think that we need to think in terms of what a language
%% provides to the programmer over an alternative language.  That is,
%% while XML is great, I am not sure that by itself it provides
%% compelling advantages over HTML.  I think that *by itself* Mace should
%% (and does) provide benefits over C/C++, Java, etc.

% 
% installing.tex : part of the Mace toolkit for building distributed systems
% 
% Copyright (c) 2011, Charles Killian, Dejan Kostic, Ryan Braud, James W. Anderson, John Fisher-Ogden, Calvin Hubble, Duy Nguyen, Justin Burke, David Oppenheimer, Amin Vahdat, Adolfo Rodriguez, Sooraj Bhat
% All rights reserved.
% 
% Redistribution and use in source and binary forms, with or without
% modification, are permitted provided that the following conditions are met:
% 
%    * Redistributions of source code must retain the above copyright
%      notice, this list of conditions and the following disclaimer.
%    * Redistributions in binary form must reproduce the above copyright
%      notice, this list of conditions and the following disclaimer in the
%      documentation and/or other materials provided with the distribution.
%    * Neither the names of the contributors, nor their associated universities 
%      or organizations may be used to endorse or promote products derived from
%      this software without specific prior written permission.
% 
% THIS SOFTWARE IS PROVIDED BY THE COPYRIGHT HOLDERS AND CONTRIBUTORS "AS IS"
% AND ANY EXPRESS OR IMPLIED WARRANTIES, INCLUDING, BUT NOT LIMITED TO, THE
% IMPLIED WARRANTIES OF MERCHANTABILITY AND FITNESS FOR A PARTICULAR PURPOSE ARE
% DISCLAIMED. IN NO EVENT SHALL THE COPYRIGHT OWNER OR CONTRIBUTORS BE LIABLE
% FOR ANY DIRECT, INDIRECT, INCIDENTAL, SPECIAL, EXEMPLARY, OR CONSEQUENTIAL
% DAMAGES (INCLUDING, BUT NOT LIMITED TO, PROCUREMENT OF SUBSTITUTE GOODS OR
% SERVICES; LOSS OF USE, DATA, OR PROFITS; OR BUSINESS INTERRUPTION) HOWEVER
% CAUSED AND ON ANY THEORY OF LIABILITY, WHETHER IN CONTRACT, STRICT LIABILITY,
% OR TORT (INCLUDING NEGLIGENCE OR OTHERWISE) ARISING IN ANY WAY OUT OF THE
% USE OF THIS SOFTWARE, EVEN IF ADVISED OF THE POSSIBILITY OF SUCH DAMAGE.
% 
% ----END-OF-LEGAL-STUFF----
\section{Getting and Installing Mace}
\label{sec:installing}

\subsection{Mace dependencies}

Mace requires the following system software packages:

\begin{description}

\item[gcc/g++] The GNU C and C++ compilers.  Mace is tested with GCC
  versions 3.4 (including MinGW), 4.0, 4.1, and 4.2.  It has in the past
  worked with a few GCC version 3.3.X systems, but most frequently
  causes internal compiler segmentation faults.  Using at least version
  3.4 is highly recommended.  This should also be accompanied with a
  build system that cmake can output builds for.

\item[perl] Perl version 5.8 or greater.  Additionally,
Class-MakeMethods and Parse-RecDescent modules are needed (and are
included in mace-extras).
%TODO: I don't think people need to get these anymore, with mace-extras,
%right?

\item[system libraries] libpthread, libm, libcrypto, libstdc++, and libssl.

\item[libboost] Used in a few places for shared pointers and lexical casting.

\item[libdb\_cxx] Version 4.2, 4.3 or 4.4.  This is the Berkeley DB
  package, and is optional for building Mace.

\item[cmake] Mace uses the \href{http://www.cmake.org}{CMake} build
  system for cross-platform building.  

\item[lgrind] For making the documentation from the source files, beautifying 
  the source code.  Also needs LaTeX.

\end{description}

Mace is regularly tested on Debian GNU/Linux and CentOS.  It has been
tested on 32 and 64 bit Linux systems, 64 bit OSX systems, and WinXP
(under MinGW). 

\paragraph{Compiling on Windows.}
When building Mace under MinGW (Windows), users will have to install
OpenSSL libraries, GNU libregex, pthreads for windows, and (if desired)
build the Berkeley DB C++ library by hand (the binary distribution does
not include it as far as we could tell).

% XXX
% Chip, what else should we write here about platforms?


\subsection{Getting the Mace source code}
\label{sec:downloading}

Mace is Open Source Software which is be published under a BSD-style license.
The latest Mace source code release can be found at the following URL: \\

\href{http://mace.ucsd.edu/release}{http://mace.ucsd.edu/release} \\

Download this file and save it in the directory where you will be
doing your development.  Unpack it using 
\command{tar -xvzf mace-latest.tar.gz}
where  latest is replaced by the version string
from the version downloaded.

Or download mace from the subversion repository using: \\

\command{svn co http://mace.ucsd.edu/svn/mace/trunk mace}

\subsection{Installing Mace}
\label{sec:unpacking}

Currently, the Mace distribution does not require the installation of any
Mace-related files outside of the Mace source tree.  All development using Mace
can be done inside the source tree.

To build the Mace distribution (in brief), use the following commands:

\begin{screen}
\command{
$ cd mace
$ mkdir builds
$ cd builds
$ cmake ../
$ make edit\_cache #or ccmake .
$ make
}
\end{screen}
%$

This will build the Mace libraries (described in
\S~\ref{sec:lib}), services
(\S~\ref{sec:packaged-services}), and applications
(\S~\ref{sec:applications}).

\subsection{The Mace Build System}
\label{sec:cmake}

Mace uses CMake to generate its build system.  CMake greatly simplified
the process of supporting various platforms, with different paths,
operating systems, bit sizes, and build systems.  It can even generate
visual studio projects (though Mace cannot build presently under visual
studio).  

This document is far from an instruction guide to using CMake, but in
this section, we'll cover several topics about using CMake to build
Mace.  If you know CMake, or just want to get on to learning about Mace,
you can skip the rest of this section for now.

\subsubsection{CMake is a build system generator}

CMake does not actually build Mace.  What it does is generate the build
system, which in turn does the actual work of building Mace.  In my own
work with Mace, this output build system has always been Makefiles for
use with GNU Make.  Readers should refer to the CMake documentation for
other options.  References following will assume a Makefile build.

\subsubsection{Out of source builds}

Mace now requires builds to be done "out-of-source."  What this means is
that none of the generated code, object files, or application binaries
are placed inside the source directory.  Attempts to configure CMake for
in-source build will be rejected by the Mace CMakeLists.txt (CMake
configuration file).  Out of source builds have the advantage that they
do not modify the original source directory, so building multiple
outputs is simplified, and cleaning up builds is also simpler.  Also, it
avoids confusion from having generated source files along side the
original source files.

\subsubsection{Running CMake}

To run CMake, use \command{cmake [path-to-mace]}, where you supply cmake
the path to the top level Mace directory.  CMake will read the
CMakeLists.txt in that directory, and configure the build accordingly.
You should run cmake from the directory in which you intend to build
Mace.  If all goes well, it will indicate that the build system is
generated (not just configured), and there will now be a Makefile you
can use.

\subsubsection{Configuring a CMake Build (and turning on debugging symbols)}

When you run cmake, a number of things may need configuring.  This may
include the paths to various required or optional libraries, header
files, or include directories.  Alternately, you may wish to set a
release type.  There are (at least) two ways to configure a CMake build.
If the build system hasn't been generated yet, you have to use the
first.  That method is \command{ccmake [path]}, where ccmake is either
the path to a partially configured CMake build (usually '.'), or the
path to the Mace top-level directory (to start a new build directory).
This opens the configuration editor, where you can set variables and
reconfigure the build.  After setting a variable, you must first run
"Configure" (perhaps more than once), and finally "Generate", to
regenerated the build system.  Until you do this, changed values may not
be saved.  This method is probably what will be needed after a failed
CMake configuration (e.g. for a missing library).  

If the build system has already been generated, another way to enter the
configuration editor is the command \command{make edit\_cache}.  This is
the method commonly used for changing configuration parameters after a
complete configuration, such as the release type.  By default, CMake
will generated a normal build, in which no special flags are included.
Other options include "DEBUG" (enable debugging symbols),
"RELWITHDEBINFO" (enable moderate optimizations and debugging symbols),
and "RELEASE" (enable extreme optimizations and not debugging symbols).
After changing the release type, use Configure and Generate to
regenerate the build system.

\subsubsection{Changing the CMake Input Configuration Files}

If you edit any of the CMake input files, the build system will be
automatically regenerated the next time you run "make".  Be careful not
to edit any of the generated input files, as these will just be
overwritten.

\subsubsection{Adding New Source Files}

In writing our CMake input files, rather than listing each source file
manually (and editing the input files when adding new files), we use the
globbing features to have CMake scan for input files.  As a result, when
you add new files, CMake will \emph{not detect} that a change has
occurred.  To give CMake the opportunity to include the newly added
files, run \command{cmake rebuild\_cache}, which will cause the build
system to be reconfigured and generated.  (This also applies if new
files were added by subversion, so be aware you may need to run this in
that case).

\subsubsection{Features of CMake Makefiles}

CMake Makefiles have some automatically built features.  These include:
\begin{description}
\item [progress] Percentages are displayed during builds, which give a
rough idea of how far through the build you are.  (These are rough
only---in my experience they may halt either before or after 100\%.  In
one rare circumstance, I saw it get as high as 500\%.
\item [quiet] By default, the builds are quiet, displaying descriptions
of what is going on.  To see the full commands, include "VERBOSE=1" in
the command, such as \command{make VERBOSE=1}.
\item [parallel build] CMake Makefiles support parallel builds.  To run
make with 4 parallel processes, use \command{make -j4}.  
\item [help] From any level of the Makefile, you can run \command{make
help}, which displays all targets you could make.
\item [edit\_cache] \command{make edit\_cache} invokes the CMake configuration editor.
\item [rebuild\_cache] \command{make rebuild\_cache} re-runs cmake to
regenerate the build system.
\item [selective build] When building any particular application or
service, CMake will only build (or rebuild) those services which are
needed (instead of always building all services).
\item [shared libraries] CMake can build Mace libraries as shared
libraries instead of static libraries, by editing the configuration to
set USE\_SHARED to on.  Note that this is not as well tested, and may
not work in all circumstances.
\end{description}

\input{simple-lg}
\input{firstpingdetail-lg}
\input{ping-lg}
\input{pingdetail-lg}
\input{real-lg}
\input{advanced-lg}
\input{faq-lg}
\input{appendix-lg}

\end{document}
