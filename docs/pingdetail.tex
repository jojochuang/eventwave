% 
% pingdetail.tex : part of the Mace toolkit for building distributed systems
% 
% Copyright (c) 2007, Charles Killian, Dejan Kostic, Ryan Braud, James W. Anderson, John Fisher-Ogden, Calvin Hubble, Duy Nguyen, Justin Burke, David Oppenheimer, Amin Vahdat, Adolfo Rodriguez, Sooraj Bhat
% All rights reserved.
% 
% Redistribution and use in source and binary forms, with or without
% modification, are permitted provided that the following conditions are met:
% 
%    * Redistributions of source code must retain the above copyright
%      notice, this list of conditions and the following disclaimer.
%    * Redistributions in binary form must reproduce the above copyright
%      notice, this list of conditions and the following disclaimer in
%      the documentation and/or other materials provided with the
%      distribution.
%    * Neither the names of Duke University nor The University of
%      California, San Diego, nor the names of the authors or contributors
%      may be used to endorse or promote products derived from
%      this software without specific prior written permission.
% 
% THIS SOFTWARE IS PROVIDED BY THE COPYRIGHT HOLDERS AND CONTRIBUTORS "AS IS"
% AND ANY EXPRESS OR IMPLIED WARRANTIES, INCLUDING, BUT NOT LIMITED TO, THE
% IMPLIED WARRANTIES OF MERCHANTABILITY AND FITNESS FOR A PARTICULAR PURPOSE ARE
% DISCLAIMED. IN NO EVENT SHALL THE COPYRIGHT OWNER OR CONTRIBUTORS BE LIABLE
% FOR ANY DIRECT, INDIRECT, INCIDENTAL, SPECIAL, EXEMPLARY, OR CONSEQUENTIAL
% DAMAGES (INCLUDING, BUT NOT LIMITED TO, PROCUREMENT OF SUBSTITUTE GOODS OR
% SERVICES; LOSS OF USE, DATA, OR PROFITS; OR BUSINESS INTERRUPTION) HOWEVER
% CAUSED AND ON ANY THEORY OF LIABILITY, WHETHER IN CONTRACT, STRICT LIABILITY,
% OR TORT (INCLUDING NEGLIGENCE OR OTHERWISE) ARISING IN ANY WAY OUT OF THE
% USE OF THIS SOFTWARE, EVEN IF ADVISED OF THE POSSIBILITY OF SUCH DAMAGE.
% 
% ----END-OF-LEGAL-STUFF----
\section{Mace Specification Revisited}
\label{sec:pingdetail}

In this section, we will discuss the changes we made for the revised
Ping implementation.


\subsection{Services (Revised)}

We changed the default route service to a UDP
\classname{RouteServiceClass}.  We do not need the reliable byte
stream semantics of TCP, and a UDP route service has less overhead.

\begin{programlisting}
services {
  Route router = UDP();
  //   Route router = TcpTransport();
} // services
\end{programlisting}

Note that we do not have to change \emph{any} other code, other than
the ``= UDP()'', to switch between TCP and UDP route
services.  This illustrates a key advantage of using service class
interfaces: because both the \classname{TcpTransport} and
\classname{UDPService} provide the \classname{RouteServiceClass}, our
code will compile seamlessly with either of them.

\subsection{Constructor Parameters}

The \symbolkw{constructor\_parameters} block allows you to list variables that
can be initialized as parameters in the service constructor.  These values
must be given defaults in the parameter list for the constructor.

\begin{programlisting}
constructor_parameters {
  uint PING_INTERVAL = DEFAULT_PING_INTERVAL;
  double PING_TIMEOUT = DEFAULT_PING_TIMEOUT;
} // constructor_parameters
\end{programlisting}

We make \variablename{PING\_INTERVAL} and \variablename{PING\_TIMEOUT}
constructor parameters, so that they can be set at runtime, as opposed
to being fixed as constants of the Ping service.  We also provide
defaults, which are defined in the \symbolkw{constants} block.


\subsection{States}

The \symbolkw{states} block allows you to list the possible states that your
service could assume\footnote{Recall that services are modeled as State
Machines.  These states are the high-level states of the finite automata.}.
All services have an implicit state named \literal{init}.  You can change the
state of your service by assigning the new state (one of those declared in the
\symbolkw{states} block) to the special Mace variable \variablename{state},
which is defined for each service instance.  You can also reference your
current state through this variable to compare it to any of the states
declared in the \symbolkw{states} block, plus \variablename{init}.  We will
see examples a little later.

\begin{programlisting}
states {
  ready;
} // states
\end{programlisting}

Our Ping service defines just one new state, \variablename{ready}.  We will
use this state to prevent many methods from being called on our service before
the \function{maceInit} method is called.  Note that qualifying the maceInit
function with the guard expression is not strictly necessary, the generated
code treats the \function{maceInit} specially and only allows it to be called
once.  This is also true of the \function{maceExit} function.


\subsection{Typedefs}

The \symbolkw{typedefs} block allows you to alias types needed for your
service with the normal C++ \symbolkw{typedef} command.  The primary purpose
for this block is to allow you to create typedefs that can both use types
defined in the \symbolkw{auto\_types} block (not used in this example) in the
typedef and use typedefs in that and the \symbolkw{messages} block.

\begin{programlisting}
typedefs {
  typedef mace::list<uint64_t> PingTimeList;
  typedef mace::hash_map<MaceKey, PingTimeList> PingListMap;
  typedef mace::hash_map<MaceKey, uint64_t> TimeMap;
  typedef mace::multimap<MaceKey, registration_uid_t> RegistrationMap;
} // typedefs
\end{programlisting}

Ping defines four new types: \typename{PingTimeList}, which we will use to
store a list of times which unacknowledged ping messages were sent;
\typename{PingListMap}, which we will use to associate a
\typename{PingTimeList} with each host; \typename{TimeMap}, which we will use
to store the last time that we received a \typename{PingReply} message from a
host; and a \typename{RegistrationMap}, which will allow us to store which
handlers (as indicated by their registration id) are interested in callbacks
for which hosts.

\subsection{Messages (Revised)}

Since in our new ping service, we will send multiple pings to each host, 
it may be necessary to distinguish which one a given reply corresponds to.
A simple way of accomplishing this is to pass the time the ping is sent
to the recipient and having them echo this back to the sender.  Thus we add
an additional field to each message.

\begin{programlisting}
messages {
  Ping { 
    uint64_t t;
  }
  PingReply {
    uint64_t techo;
    uint64_t t;
  }
} // messages
\end{programlisting}

\subsection{State Variables (Revised)}

\bigskip
\begin{programlisting}
state_variables {
  PingListMap sentTimes;
  TimeMap responseTimes;
  RegistrationMap rids;
  timer sendtimer __attribute((recur(PING_INTERVAL)));
} // state_variables
\end{programlisting}

We have changed our state variables to include the three maps,
mentioned above.  We still have the \typename{timer}
\variablename{sendtimer}, as before.  However, we use optional Mace
syntax to specify that it is a recurring timer.  That is, the timer
will automatically reschedule itself to expire again after
\variablename{PING\_INTERVAL} microseconds.


\subsection{Transitions (Revised)}

Our transitions have been revised and are somewhat more involved.  You will
notice that they now make use of the optional state expressions.  These can
occur either immediately before the transition name, in parentheses, or they
can appear as a block around a set of transitions, with the state expression
in parentheses preceding the opening of the block.  State expressions may be
arbitrary boolean expressions, with the restriction that they may not cause
side-effects or modify any of the service's state.  State expressions may
reference any of the service's state variables and global variables
(constants, constructor parameters, and \variablename{state}).  A transition
will only be executed if its state expression evaluates to \literal{true}.  If
no state expression is provided, \literal{(true)} is assumed (meaning of
course that the transition should always be executed).  However, for any given
event, only the first matching transition with a true state expression will be
executed.

\begin{programlisting}
  downcall (state == init) maceInit() {
    state = ready;
    sendtimer.schedule(PING_INTERVAL);
  } // maceInit
\end{programlisting}

Here we define the \function{maceInit()} API method, overriding the
empty default implementation in \classname{ServiceClass}.  The state
expression \function{(state == init)} ensures that
\function{maceInit()} will only be executed if our service is in the
\variablename{init} state.  Calling \function{maceInit()} in any other
state (such as \variablename{ready}) will result in a \emph{NOP}.

\function{maceInit()} does two things: 1) it changes the service's
state to \variablename{ready}, and 2) it schedules the timer.  Because
the timer is recurring, it only needs to be scheduled once---it will
continue to fire every \function{recur()} interval until it is
canceled.


\begin{programlisting}
  (state == ready) {
    ...
  } // state == ready
\end{programlisting}

This is the second form of state expression syntax, which indicates
that all transitions within the enclosing block should only be
executed if we are in the \variablename{ready} state.  This prevents
them from being called before \function{maceInit}.


\begin{programlisting}
    upcall deliver(const MaceKey& src, const MaceKey& dest, const PingReply& msg) {
      if(sentTimes[src].empty() || sentTimes[src].front() > msg.techo) {
        return; // ping already notified as failure
      }
      while(!sentTimes[src].empty() && sentTimes[src].front() <= msg.techo) {
        sentTimes[src].pop_front();
      }
      responseTimes[src] = curtime;
      if(curtime-msg.techo <= PING_TIMEOUT) {
        upcallSuccess(src, msg.techo, responseTimes[src], msg.t);
      } else {
        upcallFailure(src, msg.techo);
      }
    } // deliver PingReply
\end{programlisting}

The \function{deliver()} callback for the receipt of a \typename{PingReply}
message has the same idea as in our previous Ping service, but it has to do
some bookkeeping.  First, we check if the ping has already been timed out in
the sendtimer transition, and exit if it has.  If not, we update the
unacknowledged ping list, and then update the response time for the given host.
Then, if the ping reply was received before the timeout we use the routine
\function{upcallSuccess()}, defined in the \symbolkw{routines} block (below),
to notify interested handlers of the result. Otherwise, we use the
\function{upcallFailure()} routine to notify interested handlers of the
negative result.

\begin{programlisting}
    downcall monitor(const MaceKey& host, registration_uid_t rid) {
      if (rids.find(host) == rids.end()) {
        ping(host);
      }
      rids.insert(std::make_pair(host, rid));
    } // monitor
\end{programlisting}

We update the \function{monitor()} API method as follows.  We check if
we are already monitoring this host.  If not (the condition checked in
the \symbolkw{if}), then call the routine \function{ping()}.  This
check is needed to suppress sending duplicate \typename{Ping} messages
to a host when different handlers request to monitor the same host.
We also need to insert the mapping from the host to the registration
id in our registration map, which will allow us to make callbacks on
the interested handler.


\begin{programlisting}
    downcall getLastResponseTime(const MaceKey& host) {
      return responseTimes[host];
    } // getLastResponseTime
\end{programlisting}

We implement the \function{getLastResponseTime} API method as a
straight-forward lookup in our map.


\begin{programlisting}
  scheduler sendtimer() {
    for (PingListMap::iterator i = sentTimes.begin(); i != sentTimes.end(); i++) {
      if(!i->second.empty() && curtime - i->second.front() > PING_TIMEOUT) {
	upcallFailure(i->first, i->second.front());
        while(!i->second.empty() && curtime - i->second.front() > PING_TIMEOUT) {
          i->second.pop_front();
        }
      }
      ping(i->first);
    }
  }
\end{programlisting}

Because our service monitors multiple hosts, and does so indefinitely, our
timer transition has more complexity.  The timer is responsible for two main
tasks: 1) notifying timeouts (failures) and 2) sending the next
\typename{Ping} message.  Both of these tasks must be performed on each host
that we are monitoring.  Thus, we loop over the \variablename{sentTimes} map,
which returns us a (host, PingTimeList) pair, for every host that we are
monitoring.  

We need to make a failure callback when the PingTimeList (which is storing
the times of unacknowledged pings) is non-empty, and the first element
of the list is more than \variablename{PING\_TIMEOUT} microseconds in the 
past.  After we notify the failure, we remove all unacknowledged ping messages
which are expired from the list.

Finally, we send our next ping to the host, regardless of whether we
think it is live or not.


\subsection{Routines}

Routines are simply C++ methods that are not part of the service API
(as either downcalls or callbacks).  They are defined within the
\symbolkw{routines} block.  Service routines should be defined for the
same reason you would define a normal class method: to eliminate code
duplication for common procedures, or to abstract away a complicated
piece of code, etc.


\begin{programlisting}
  void ping(const MaceKey& host) {
    downcall_route(host, Ping(curtime));
    sentTimes[host].push_back(curtime);
  } // ping
\end{programlisting}

The \function{ping()} method sends a \typename{Ping()} message to the
host, and records the time in our sent times map.

\begin{programlisting}
  void upcallSuccess(const MaceKey& k, uint64_t ts, uint64_t tr, uint64_t rt) {
    for (RegistrationMap::iterator i = rids.find(k); i != rids.end() && i->first == k; i++) {
      upcall_hostResponseReceived(k, ts, tr, rt, i->second);
    }
  } // upcallSuccess

  void upcallFailure(const MaceKey& k, uint64_t ts) {
    for (RegistrationMap::iterator i = rids.find(k); i != rids.end() && i->first == k; i++) {
      upcall_hostResponseMissed(k, ts, i->second);
    }
  } // upcallFailure
\end{programlisting}

\function{upcallSuccess()} and \function{upcallFailure()} are helper
methods that ensure that we make the appropriate callback on all
interested handlers.  In both methods, we loop over our registration
map, which contains (host, registration-id) pairs.

