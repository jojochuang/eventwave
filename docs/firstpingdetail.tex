% 
% firstpingdetail.tex : part of the Mace toolkit for building distributed systems
% 
% Copyright (c) 2011, Charles Killian, Dejan Kostic, Ryan Braud, James W. Anderson, John Fisher-Ogden, Calvin Hubble, Duy Nguyen, Justin Burke, David Oppenheimer, Amin Vahdat, Adolfo Rodriguez, Sooraj Bhat
% All rights reserved.
% 
% Redistribution and use in source and binary forms, with or without
% modification, are permitted provided that the following conditions are met:
% 
%    * Redistributions of source code must retain the above copyright
%      notice, this list of conditions and the following disclaimer.
%    * Redistributions in binary form must reproduce the above copyright
%      notice, this list of conditions and the following disclaimer in the
%      documentation and/or other materials provided with the distribution.
%    * Neither the names of the contributors, nor their associated universities 
%      or organizations may be used to endorse or promote products derived from
%      this software without specific prior written permission.
% 
% THIS SOFTWARE IS PROVIDED BY THE COPYRIGHT HOLDERS AND CONTRIBUTORS "AS IS"
% AND ANY EXPRESS OR IMPLIED WARRANTIES, INCLUDING, BUT NOT LIMITED TO, THE
% IMPLIED WARRANTIES OF MERCHANTABILITY AND FITNESS FOR A PARTICULAR PURPOSE ARE
% DISCLAIMED. IN NO EVENT SHALL THE COPYRIGHT OWNER OR CONTRIBUTORS BE LIABLE
% FOR ANY DIRECT, INDIRECT, INCIDENTAL, SPECIAL, EXEMPLARY, OR CONSEQUENTIAL
% DAMAGES (INCLUDING, BUT NOT LIMITED TO, PROCUREMENT OF SUBSTITUTE GOODS OR
% SERVICES; LOSS OF USE, DATA, OR PROFITS; OR BUSINESS INTERRUPTION) HOWEVER
% CAUSED AND ON ANY THEORY OF LIABILITY, WHETHER IN CONTRACT, STRICT LIABILITY,
% OR TORT (INCLUDING NEGLIGENCE OR OTHERWISE) ARISING IN ANY WAY OUT OF THE
% USE OF THIS SOFTWARE, EVEN IF ADVISED OF THE POSSIBILITY OF SUCH DAMAGE.
% 
% ----END-OF-LEGAL-STUFF----
\section{Mace Specification Overview: Ping in Detail}
\label{sec:firstping-detail}

This section explains the various parts of Ping service
implementation.

\subsection{Header Files}
\label{sec:header-files}

All code between the start of a \mac file and the first recognized
Mace keyword is copied verbatim as C++ code into the generated header
file in your service implementation.  Thus, you should include
whatever standard library header files you need at the top of your
\mac file.

You do not, however, need to include header files for mace services or
handlers that you reference.  The Mace compiler will generate include
statements for all the service classes and handlers used by your
service, so it is not necessary for you to include these yourself.
For instance, even though \filename{FirstPing.mac} needs to reference the
generated \filename{PingServiceClass.h} and
\filename{PingDataHandler.h} files, we can omit the
\symbolkw{\#include} directives for these files because the compiler
will generate them for us.


\subsection{Service}
\label{sec:service}

The beginning of our example includes a ``Service'' statement.

\begin{programlisting}
service FirstPing;
\end{programlisting}

The service statement specifies the short name of the service.  The long name
is generated from the short name by appending ``Service'' to it, to make
\classname{PingService}.  The build system provided with Mace requires the
service line to match the filename.  As a shortcut, this statement may be
abbreviated as:

\begin{programlisting}
service;
\end{programlisting}

This abbreviated form will continue to work if the filename is changed.  Future
versions of Mace will feature a different syntax for the service name.

\subsection{Provides}
\label{sec:provides}

The beginning of our example includes a ``Provides'' statement.

\begin{programlisting}
provides Ping;
\end{programlisting}

The provides statement specifies the service class of our Ping
service.  This line causes the Mace compiler to read the
\classname{PingServiceClass} interface file (which we also wrote), and
generate C++ code for the Ping service so that it inherits from
\classname{PingServiceClass}.

The syntax of this statement is simple: the keyword
\symbolkw{provides}, the name of the service class (with the suffix
\classname{ServiceClass} omitted), and a semicolon.  Omitting this
line causes your service to only provide the ``Null'' service class,
which is defined just as the initialization and exit functions.


\subsection{Services}
\label{sec:services}

After indicating the interface of the Ping service, we next specify the set of
services we use.  In the case of Ping, we mainly need a service which can
route messages for us over the network to a destination address.  This
functionality exists as part of the \classname{RouteServiceClass} interface,
and there is an included TcpTransport service which implements it.

To tell Mace we want to use this service, we put it in the services block:

\begin{programlisting}
services {
  Route router = TcpTransport();
} // services
\end{programlisting}

This creates a variable named \variablename{router}, a reference to an
object providing the \classname{Route} service class (defined in
\classname{RouteServiceClass}).  We also specify that, by default,
this should be constructed as a new TCP service.


\subsection{The Route Service Class (briefly)}
\label{sec:route-service-overview}

The \classname{Route} service class defines the \function{route}
method for sending data:

\begin{programlisting}
bool route(const MaceKey& dest, const std::string& msg, registration_uid_t regId);
\end{programlisting}

This method is called with a destination address and a buffer of data
to send, as well as a registration id.  Delivery of data is made to
the appropriately registered \classname{ReceiveDataHandler} using its
\function{deliver} method with a \variablename{commType} of
\literal{COMM\_TYPE\_UNICAST}:

\begin{programlisting}
void deliver(const MaceKey& source, const MaceKey& dest, const std::string& msg, 
             registration_uid_t regId);
\end{programlisting}

% XXX
% mention how we handle errors
We have now covered the basics necessary to get going.  The TcpTransport
service, which provides this interface, handles the management and
connection of sockets as well as properly catching errors for you,
isolating you from this burden (you can register a
\classname{NetworkErrorHandler} to receive error notifications if
necessary).  To send a message to a peer, just call \function{route}.
When the message gets to the remote destination, the TCP service calls
\function{deliver} on the registered data handler with the same
message buffer.


\subsection{Under The Covers}
\label{sec:route-working-details}

When you specify that you will use a \classname{Route} service class
in \filename{FirstPing.mac}, the compiler automatically makes the Ping
service a \classname{ReceiveDataHandler}, and when a Ping service is
initialized, it handles initializing the RouteServiceClass instance it
will use, constructing one if necessary, and then registering itself
with that RouteServiceClass instance.  The \variablename{router}
variable, defined in the services block, \emph{actually} refers to the
registration id obtained from that registration, not to the service
instance.  To utilize Ping's ``router'', instead of calling methods
directly on it, you call the methods of it through helper functions
with the same parameter list but a modified name.  For example, to
call \function{route}, you actually call \function{downcall\_route}.
The ``downcall\_'' refers to the fact that the router is a service you
are using (and therefore lower in the service stack).  We saw the use of
``downcall\_'' both in Ping's transitions for sending data and in the
\symbolkw{method\_remappings} block, described below.
%Then, by passing in the \variablename{router} variable as the
%registration id parameter the generated code handles making the route
%call on the appropriate service.  Upcalls are callback on handlers
%registered with your service, and we shall see their use a little
%later.


\subsection{Constants}
\label{sec:constants}

Next we define a constant variable which the Ping service will use to
specify the maximum time to wait for a response.

\begin{programlisting}
constants {
  uint64_t PING_TIMEOUT = 2 * 1000 * 1000; // 2 seconds in micros
} // constants
\end{programlisting}

Defining constants is as easy as creating a block for them, and then defining
them using normal syntax.  Note that you do not use the C++ keyword
\symbolkw{const} when specifying service constants.  Now you can use these
constant variables anywhere else---like in transition code, array sizes and
template parameters.

\subsection{Messages}
\label{sec:firstmessages}

Now that we have discussed how the calls are mapped to handle
messages, we will discuss how you actually define messages.  Ping
defines two messages, one for the request sent to the peer, and one
for the response.

\begin{programlisting}
messages {
  Ping { }
  PingReply {
    uint64_t t;
  }
} // messages
\end{programlisting}

The first message, ``Ping'', has no data fields associated with it.
The \function{deliver} method already tells us who sent the data, so
there is no need to include the sender as a part of the message.  The
second message, ``PingReply'', has one field---an unsigned 64-bit
integer named ``t''.

\subsection{State Variables}
\label{sec:state-variables}

The state variables\footnote{Mace services are modeled as finite automata.
Thus, the variables defined in the \symbolkw{state\_variables} block record
persistent state information about the service.} are like member variables in
a class.  They are specific to the running instance of a Ping service, and
they are accessible anywhere within the service specification.  For the most
part they are defined like normal C++ variables.

\begin{programlisting}
state_variables {
  MaceKey target;
  uint64_t timeSent;
  timer sendtimer;
} // state_variables
\end{programlisting}

For the Ping service, we define three variables, one for the
destination address of the ping, one for the time we sent the ping,
and one for the timer we use to cancel the ping if it takes too long.

\subsection{Transitions}
\label{sec:transitions}

The \symbolkw{transitions} block contain methods that the service
executes in response to events.  Each method is preceded by a keyword
specifying the transition type and an optional state expression
%   (state
    %expressions are discussed in a later section)
.  There are three types
of transition methods:

\begin{description}

\item[\symbolkw{upcall}] This method will be called by one of the
  service instances listed in the \symbolkw{services} block that your
  service uses.  Upcalls are the way lower-layer services signal
  events to higher-layer services (such as delivering a message).
  Upcall transitions must be defined by a handler of a service that
  your service uses.

\item[\symbolkw{downcall}] This method represents an API call that can
  be made on your service from either applications or services that
  \emph{use} your service.  Downcall transitions must be defined by a
  service class that your service provides.

\item[\symbolkw{scheduler}] This method will be called by the central
  Mace runtime library scheduler in response to a timer event.
  Scheduler methods are the same as the names of the declared timers.
  Each timer calls its transition when it expires.

\end{description}

The Ping service defines four transitions: two for delivering messages
(one for each message type defined in the \symbolkw{messages} block),
one to handle the monitor API call, and one to handle the sendtimer.
Each transition is described below.


\begin{programlisting}
  upcall deliver(const MaceKey& src, const MaceKey& dest, const Ping& msg) {
    downcall_route(src, PingReply(curtime));
  } // deliver Ping
\end{programlisting}

The first transition, an upcall, is called on our service by our
\classname{RouteServiceClass} \variablename{router} when it receives a
\typename{Ping} message.  (The prototype for \function{deliver} is defined in
\classname{ReceiveDataHandler}, which is listed as one of the
\symbolkw{handlers} for \classname{RouteServiceClass}.)  When our Ping service
receives a \symbolkw{Ping} message, then we want to send a
\symbolkw{PingReply} message in response.  We do this by making a
\function{downcall\_route} call on our \variablename{RouteServiceClass}
(recall that the object (\variablename{router}) on which the method is being
invoked, normally passed as the final argument, is automatically inferred).
We pass in the destination for the message, \variablename{src}, the host that
sent us the \typename{Ping} message; and we pass the message we want to route,
a \typename{PingReply}, which is constructed with the current time (in
microseconds), which is available throughout Mace services using the variable
``curtime''.


\begin{programlisting}
  upcall (src == target) deliver(const MaceKey& src, const MaceKey& dest, const PingReply& msg) {
    sendtimer.cancel();
    upcall_hostResponseReceived(src, timeSent, curtime, msg.t);
  } // deliver PingReply
\end{programlisting}

The second transition, another deliver upcall, defines what action we take
when we receive a \typename{PingReply} message.  This transition has been
guarded by a state expression, which in this case verifies that the source of
the \typename{PingReply} message is the same as the \variablename{target}
being pinged.  If ping replies are received from erroneous destinations, no
action will be taken.  In this transition we first cancel our timer, because
we have received a response.  Then we make an upcall to the application or
service using us, to report that we received a response.  Recall that we
defined \function{hostResponseReceived} in \classname{PingDataHandler}.
Notice that you access message fields exactly as you would class member
fields, such as ``msg.t''.


\begin{programlisting}
  downcall monitor(const MaceKey& host) {
    timeSent = curtime;
    target = host;
    downcall_route(target, Ping());
    sendtimer.schedule(PING_TIMEOUT);
  } // monitor
\end{programlisting}

The third transition, a downcall, overrides the default method
implementation for \function{monitor} specified in
\classname{PingServiceClass}.  When an application or service requests
that we monitor a host, we do the following: 1) store the current
time, so that we can report it when we make a success or failure
upcall; 2) store the target, so that we can report it on failure and use
it to protect the delivery of erroneous ping replies; 3)
send a new \typename{Ping} message to the target, using our
\classname{RouteServiceClass} \variablename{router}; and 4) schedule
our timer to expire after \variablename{PING\_TIMEOUT} microseconds.


\begin{programlisting}
  scheduler sendtimer() {
    upcall_hostResponseMissed(target, timeSent);
  }
\end{programlisting}

The final transition, of type scheduler, will fire if the timer expires.
Because we schedule the timer when we send a \typename{Ping} message in
\function{monitor}, and cancel it when we receive a \typename{PingReply}
response in \function{deliver}, the timer will only ever fire if we do not
receive a response within \variablename{PING\_TIMEOUT} microseconds.  If the
timer does fire, then we make an upcall on the application or service using us
to report that we missed an expected response.  We defined
\function{hostResponseMissed} as the other method in
\classname{PingDataHandler}.
