% 
% intro.tex : part of the Mace toolkit for building distributed systems
% 
% Copyright (c) 2007, Charles Killian, Dejan Kostic, Ryan Braud, James W. Anderson, John Fisher-Ogden, Calvin Hubble, Duy Nguyen, Justin Burke, David Oppenheimer, Amin Vahdat, Adolfo Rodriguez, Sooraj Bhat
% All rights reserved.
% 
% Redistribution and use in source and binary forms, with or without
% modification, are permitted provided that the following conditions are met:
% 
%    * Redistributions of source code must retain the above copyright
%      notice, this list of conditions and the following disclaimer.
%    * Redistributions in binary form must reproduce the above copyright
%      notice, this list of conditions and the following disclaimer in
%      the documentation and/or other materials provided with the
%      distribution.
%    * Neither the names of Duke University nor The University of
%      California, San Diego, nor the names of the authors or contributors
%      may be used to endorse or promote products derived from
%      this software without specific prior written permission.
% 
% THIS SOFTWARE IS PROVIDED BY THE COPYRIGHT HOLDERS AND CONTRIBUTORS "AS IS"
% AND ANY EXPRESS OR IMPLIED WARRANTIES, INCLUDING, BUT NOT LIMITED TO, THE
% IMPLIED WARRANTIES OF MERCHANTABILITY AND FITNESS FOR A PARTICULAR PURPOSE ARE
% DISCLAIMED. IN NO EVENT SHALL THE COPYRIGHT OWNER OR CONTRIBUTORS BE LIABLE
% FOR ANY DIRECT, INDIRECT, INCIDENTAL, SPECIAL, EXEMPLARY, OR CONSEQUENTIAL
% DAMAGES (INCLUDING, BUT NOT LIMITED TO, PROCUREMENT OF SUBSTITUTE GOODS OR
% SERVICES; LOSS OF USE, DATA, OR PROFITS; OR BUSINESS INTERRUPTION) HOWEVER
% CAUSED AND ON ANY THEORY OF LIABILITY, WHETHER IN CONTRACT, STRICT LIABILITY,
% OR TORT (INCLUDING NEGLIGENCE OR OTHERWISE) ARISING IN ANY WAY OUT OF THE
% USE OF THIS SOFTWARE, EVEN IF ADVISED OF THE POSSIBILITY OF SUCH DAMAGE.
% 
% ----END-OF-LEGAL-STUFF----
\section{Why are you reading this document?}
\label{sec:intro}

Perhaps you stumbled on this document on the web somewhere, perhaps
you have already downloaded Mace and are trying to figure out how to
use it, perhaps you are considering whether to use Mace, or perhaps
you've been told to use Mace by a boss or instructor, and need to
figure out what that means.

This document will explain what Mace is, go through a simple example
of how to use it, and discuss some of its advanced features as well as
address common concerns and questions.

\subsection{What is Mace?}
\label{sec:what-is-mace}

If you found this document looking for information on various forms of
bludgeoning weapons or self-defense sprays, then this paragraph will
be all that is in this document about your search.  None of these are
the context in which Mace will be discussed in this document.  The
name Mace is instead a derivative from the MACEDON parent project,
abbreviated to suggest the broader scope to which it is applicable.

Mace is a software package for building distributed systems.  It
builds upon the ideas from its parent project, MACEDON, by broadening
the scope of what can be designed with it, and by removing many
limitations of the original system.

Mace includes a compiler that translates service specifications into
C++ code, libraries designed to be linked together with generated
services, a distribution of existing services ready to be used by
other services or applications, and a few basic applications to run
the services contained within.

Mace seeks to transform the way distributed systems are built by
providing designers with a simple method for writing complex but
correct and efficient implementations of distributed systems.  To that
end, we are always considering new libraries and language features
which could be used to make building, designing, debugging, or
verifying distributed systems more powerful, flexible, simple, or
natural.

\subsection{Why would I want to use a domain-specific language and toolkit?}
\label{sec:why-new-toolkit}

Over the past few years, we have set out to evaluate a variety of
techniques for building robust, high-performance distributed systems.
One of our explicit goals has been to conduct head-to-head comparisons
of algorithms proposed in the literature, including our own.  Target
systems include application-layer multicast, distributed hash trees,
peer-to-peer indexing systems, and overlay routing.  One of the
primary lessons of this work thus far has been the difficulty of
translating elegant and seemingly simple distributed algorithms into
operational, high-performance, and robust systems.

We found significant commonality among the various
implementations---for instance, event handling, timer management,
failure detection, asynchronous communication, node join/departure, and
message serialization/deserialization.  However, differences in
individual application structure and implementation forced us to
repeatedly reimplement the same functionality, typically with a
different set of errors and inefficiencies each time.  Even once the
baseline system was operational, the resulting application often
performed significantly slower than expected and bugs in corner cases
would remain dormant or masked for long periods of time.

We concluded that a significant impediment to rapid prototyping of our
target applications was the lack of an appropriate development
environment for distributed applications.  Based on our experience, we
identified the following requirements to ease this development effort.

%\begin{itemize}
%\item Constructing distributed systems would be simplified by the
%  ability to compose simple distributed computing primitives into more
%  complex behavior.  
\emph{Constructing distributed systems would be simplified by the
ability to compose simple distributed computing primitives into more
complex behavior.} For instance, many distributed applications would
benefit from failure detection, consensus, multicast, barriers, and
key-based routing.  However, without well-defined API's, it is
difficult to reuse implementations or to leverage the benefits of an
improved implementation of a given logical subsystem.  In this paper,
we describe initial efforts to define required API's to support
complex, multi-layer distributed systems.

  %\item Current programming languages are not well suited to the
  %  requirements of distributed systems.  

\emph{Current programming languages are not well suited to the
requirements of distributed systems.} While there are communication
libraries and class hierarchies in languages such as C++, Java, and
Python, they typically target client/server communications (e.g., HTTP
or XMLRPC) and still provide relatively primitive support for failure
detection and recovery.  Further, we observe that the higher-level
structure of many distributed systems is logically event- and
state-based.  Each node maintains some state that may be modified as a
result of a series of events, typically message reception and timer
expiration.  Individual nodes respond to events by modifying their
state and perhaps transmitting their own message to one or more
destinations.  While this high level structure is simple to describe,
it is error prone to implement.  Further, managing asynchrony still
remains a challenge.  Delivering high performance often requires
careful consideration of appropriate locking primitives, ensuring that
individual operations do not block, and assigning the appropriate
number of threads to handle logically concurrent tasks.  Of course,
all of this can be programmed in existing languages such as C++ and,
to a lesser extent, Java. Providing the appropriate language
primitives can both significantly simplify the code and reduce
opportunities for errors.

  %\item Debugging support for distributed systems remains primitive,
  %  often reducing to custom scripts run over log files generated at
  %  hundreds of individual sites.  

\emph{Debugging support for distributed systems remains primitive,
often reducing to custom scripts run over log files generated at
hundreds of individual sites.}  Generating the necessary debugging
information can make the underlying code less efficient, less
readable, and more error prone.  With appropriate language support,
the appropriate logging information can be generated
semi-automatically.  Further, the information can be efficiently
stored in a format amenable to inserting into a SQL database.  We have
found this log of per-event system activity along with separate
programmer annotations for their \emph{expectations} of how the
distributed system should behave and \emph{assertions} about global
correctness conditions (both of which must necessarily simultaneously
apply to state stored at multiple nodes across at multiple time
granularities) to be invaluable in debugging distributed system
behavior.  These distributed expectations and assertions may apply to
to both system performance and distributed system structure.
%\end{itemize}
 
We are building Mace to be just the support that a designers and developers
need when building distributed systems.  It is a complete redesign and rewrite
of its parent project, MACEDON, targeted at a broader range of distributed
systems, and with better support for the programmer with fewer limitations.

\subsection{Why shouldn't I just use other previously existing languages?}
\label{sec:why-mace}

Though there are a few existing languages with similar goals, most do
not have the emphasis on building real systems as Mace does.  MACEDON
of course is very close, but suffers from limitations such as fixed
size message headers, restricted and monolithic API, non-portable 
network communications, lack of support for connecting with external
processes (such as XML-RPC or HTTP), lack of support for firewall
traversal, more primitive logging, and a single linear protocol/service
stack.

(Ed. Note: Comments on other systems forthcoming).

%Think of Mace as being for distributed systems what XSLT is
%for web documents.  Of course, you can write your web pages directly
%in the presentation language (HTML), but if you write them in a
%language which allows you to give meaning to the document (XML), they
%can be processed more effectively by automated tools, allowing them to
%be presented in different ways, analyzed for semantic meaning, or
%updated and modified.

%% av:
%% This is a great start at the answer to this question.

%% However, I think that we need to think in terms of what a language
%% provides to the programmer over an alternative language.  That is,
%% while XML is great, I am not sure that by itself it provides
%% compelling advantages over HTML.  I think that *by itself* Mace should
%% (and does) provide benefits over C/C++, Java, etc.
